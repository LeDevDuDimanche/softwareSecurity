\documentclass{article}

\usepackage[utf8]{inputenc}

\title{GRASS documentation}
\author{Simon Le Bail-Collet, Lukas Gelbmann, Björn Gudmundsson}
\date{May 2019}

\begin{document}

\maketitle

blah

\section{Protocol}

blah

\section{Architecture}

For this project we used the classic client/server architecture. The server will listen for incoming connections on a socket and when a client wants to start
a session on our platform the server will start a new thread that will open up a port and will listen for incoming commands from the client. The commands are then parsed 
and passed to a commands API. The methods stated in the API are the same as given in the specification of the GRASS protocol. The structure of the program is as follows: All of the
headers of the program are listed in a directory called include and those headers are split into server and client headers and headers that are shared between both
implementations of the server and the client. 

A client session in our implementation is identified by a connection object. The connection object is unique per user and session and keeps track of the state of
an active session. An example of how the connection object is used is in how we implemented the traversal from one directory to another. One of the instance variables 
variables in the current relative path of the session. It keeps track of where the user is located in the shared directory, relative to the base directory in the program without having to perform
cd commands on a per process basis. Every command the user issues is then calculated relative to the users current directory. 

An important issue that we had to tackle was implementing such that multiple users can be using the service concurrently and that the behaviour of the system
should remain consistent while a user is logged in. An example of such a problem would be when a user would like to issue an rm command and remove a directory 
or file from the shared repository. A user may have traversed into the directory or a child directory that another user would like to issue an rm command on.
To solve that problem there are two independent global datastructures, UserReadTable and FileDeleteTable. These datastructures are shared between all active sessions and keep track
of which directories currently have users that are either residing in one of its child directories or reading a file in one of its child directories. If an rm command also 
takes a significant amount of time then a user should not be able to either read the contents of the file or traverse into one of its child directories. For that reason we also have
the global data structure of entries that are being deleted. When a user logsout or exits from the system the connection object will make sure that the user activities and locations within the system have been cleaned 
to further prevent it from preventing future actions from clients that are currently using the system. 



\section{Development}

For the development of this protocol we used a Test driven develpment style. A lot of test cases that tested out different functionalities of the protocol and 
various edge cases and worked on the project with the tests in mind. The tests were written in python and use regular expression to determine if the outputs from
the server match the expected values that were determined by the tests.



\end{document}
